\documentclass[a4paper,12pt]{article}
\usepackage[utf8]{inputenc}
\author{Максакова Мария}
\title{Физика 2 семестр}
\usepackage[T2A]{fontenc}
\usepackage[english,russian]{babel}
\usepackage{amsmath,amssymb,amsthm}
\usepackage{graphicx}
\usepackage{array}
\usepackage{ulem}
\usepackage{hyperref}

\setlength{\parindent}{0pt}
\pdfpagewidth 8.5in
\pdfpageheight 11in

\begin{document}

\section{Электростатика}
\subsection{1   Элементарные электрические заряды и взаимодействия между ними. Принцип суперпозиции. Точечные макроскопические заряды. Закон Кулона. Системы единиц. }
\textbf{Электрический заряд} - если 2 одинаковых элементарных частицы отталкиваются с силой $~1/r^2$, то такая частица заряжена.\\
Принцип суперпозиции:\\
 - Взаимодействие между двумя частицами не изменяется при внесении третьей частицы, также взаимодействующей с первыми двумя.\\
 - Энергия взаимодействия всех частиц в многочастичной системе есть просто сумма энергий парных взаимодействий между всеми возможными парами частиц. В системе нет многочастичных взаимодействий.\\
 - Уравнения, описывающие поведение многочастичной системы, являются линейными по количеству частиц.\\
Напряженность поля системы зарядов равна векторной сумме напряженностей полей, которое создает каждый из этих зарядов в отдельности.\\
Элементарный заряд в СИ $1.6 \cdot 10^{-19}$ Кл, $\varepsilon _0 = 8,85 \cdot 10^{-12}$ Ф/м\\
\textbf{Пробный заряд} - точечный положительный заряд, который не искажает исследуемое поле, т.е. не вызывает в нем перераспределения зарядов (собственным полем пробного заряда пренебрегают).\\
Заряд макроскопического тела: $Q = (N_+ - N_-)\cdot q$, где $N_+, N_-$ - кол-во положительно и отрицательно заряженных частиц\\
Электростатическое взаимодействие между элементарными зарядами: $F = \frac{k \cdot q_1 \cdot q_2 \cdot \vec{r}}{|r|^3}$\\
Системы единиц, варианты констант:(?а оно надо кому-то?)\\
$q_e = 1; m_e = 1; \hbar = 1; c = 137$ || $q_e = 1; m_e = 1; \hbar = 1/137; c = 1$\\
В СГС $k = 1$, в СИ $k =  \frac{1}{4\cdot \pi \cdot \varepsilon _0}$\\
Сила, действующая на элементарный заряд со стороны системы элементарных зарядов: $f_{1,k} = \sum _K \frac{q_1 \cdot q_2}{|R - r_k|^3}(\vec{R} - \vec{r_k})$\\
Сила взаимодействия между двумя макроскопическими зарядами: $f_{I,K} = \sum _{I,K} \frac{q_i \cdot q_2}{|R_i - r_k|^3}(\vec{R_i} - \vec{r_k})$\\
Закон Кулона: $F_{I,K} = \frac{q_i \cdot q_k \cdot \vec{\Delta R}}{|\Delta R|^3}$\\


\subsection{2   Электростатическое поле в вакууме. Вектор напряженности электрического поля. Скалярный потенциал. Связь между электрическим полем и потенциалом. Примеры вычисления электрических полей и потенциалом простейших систем.}
\textbf{Электростатическое поле} - поле, созданное системой неизменных электрических зарядов. $\vec{E} = \frac{\vec{F}}{q} = -\nabla \varphi$; $[E] = \frac{V}{m}$\\
Сглаженное макроскопическое поле, создаваемое непрерывным распределением заряда: $E = \int \frac{dq(r) (\vec{R} - \vec{r}}{|R - r|^3}$\\
Определение электрического потенциала в точке: $\varphi (R) = \int _R ^{R_0} (\vec{E} \cdot d\vec{l})$\\
Принцип суперпозиции для потенциалов: $\varphi (\vec{R}) = \sum \varphi _i (\vec{R})$\\
Потенциал: $ \varphi = \frac{k \cdot q}{r} $\\



\subsection{3-4  Теорема Гаусса. Интегральная форма записи уравнений электростатического поля в вакууме. Примеры вычисления полей при помощи интегральных теорем. Дифференциальная форма записи уравнений для электростатического поля в вакууме. Примеры использования теорем в дифференциальной форме.}
Теорема Гаусса: $\oint \vec{E} \cdot d\vec{S} = 4 \pi k Q $\\
Поле бесконечной равномерно заряженной плоскости: $E = \frac{\sigma}{2 \varepsilon _0}$\\
Уравнения Максвелла для электростатики:\\
Интегральная форма: $\oint \vec{E} \cdot d\vec{S} = 4 \pi k Q; \oint \vec{E} \cdot d\vec{l} = 0$ Да, площадь векторная.\\ Циркуляция в электростатическом поле равна нулю.\\
Дифференциальная форма: $ \nabla \cdot \vec{E} = \frac{\rho}{\varepsilon _0};\nabla \times \vec{E} = 0$\\
//ATTENTION!!! Местами перемешаны Гауссовская и СИ


\subsection{5   Уравнение Пуассона для скалярного потенциала. Теорема единственности решения  задач электростатики}
Уравнение Пуассона: $ - \nabla ^2 \varphi = \frac{\varphi}{\varepsilon _0} (SI)  = 4 \pi \rho (Gauss)$\\
Решение уравнения Пуассона: $\Delta \varphi = \int \frac{\rho (\vec{r}}{|\vec{R} - \vec{r}|} dV$\\
\textbf{Теорема единственности решения задач электростатики:} в данной системе зарядов и n проводников существует только один
$ \phi(\vec{r}) $
, обращающийся в нуль на бесконечности и принимающий установленные значения 
$ \phi_1, \phi_2... \phi_n $
на границах проводников.
Работа по внесению равномерно заряженной сферы в толе точечного заряда: $A = \frac{Q}{4 \pi R^2} \sum \Delta S_i \cdot \varphi (\vec{R_i} )$\\
Работа по внесению точечного заряда q в точку, расположенную рядом с равномерно заряженной сферой: $A = Q \cdot \varphi _0$

\subsection{6   Электрическое поле при наличии проводников. Метод изображений. Примеры расчетов полей при наличии проводников}
Внутри проводника поля нет!\\
Диэлектрическая проницаемость: $\varepsilon = \frac{E_0}{E}$\\
Проводники изменяют структуру электрического поля, поэтому встает задача расчета поля системы зарядов и проводников. \\
Метод изображений заключается в подборе такого фиктивного заряда в толще проводника, введение которого создало бы в области границы проводника условия, идентичные данным. \\
Это похоже на построение изображения заряда в зеркале, если считать поверхность проводника зеркалом. \\
\url{http://ru.wikipedia.org/wiki/%D0%9C%D0%B5%D1%82%D0%BE%D0%B4_%D0%B8%D0%B7%D0%BE%D0%B1%D1%80%D0%B0%D0%B6%D0%B5%D0%BD%D0%B8%D0%B9}
Граничные условия на поверхности проводника, выполнение которых является критерием правильности решения задачи методом изображений: $\varphi = const; E_R = 0$\\
Сила, действующая на элемент поверхности заряженного проводника: $\delta \vec{F} = \vec{E}' \delta q = \vec{E}' \sigma \delta S$\\
Электростатическое давление на поверхность заряженного проводника: $p = \frac{\delta F}{\delta S} = a \pi \sigma ^2$\\

\subsection{7   Простейшие механизмы поляризации диэлектриков: электронная и ориентационная поляризация. (Анизотропные молекулы в электрическом поле).}
Для начала скажем кое-что о диполях и дипольном моменте. \\
\textbf{Электрический диполь} - система из 2 разноименных зарядов $ q $ и $ -q $, находящихся на расстоянии $ l $ друг от друга. \\
\textbf{Дипольный момент} вычисляется по следующей формуле: $\vec{p} = q \vec{l}$\\
Потенциал диполя: $\varphi = \frac {k p} {r} \cos {\alpha}$, где $ \alpha $ - угол между $ \vec{l} $ и $ \vec{r} $. \\
Напряженность поля диполя: $\vec{E} = \frac {k p} {r^3} \sqrt {1 - 3 \cos^2 {\alpha}}$\\
В однородном поле сила, действующая на диполь, равна нулю. $\vec{F} = \vec{F_-} + \vec{F_+} = 0$\\
В неоднородном это не так:
\begin{equation}
\begin{gathered}
	\vec{F} = \vec{F_-} + \vec{F_+} = q (\vec{E_-} + \vec{E_+}) = q \Delta \vec{E} \\
	\Delta \vec{E} = \frac {\Delta E \vec{l}}{l} = \frac {\partial E} {\partial l} \vec{l} \Rightarrow \\
	\vec{F} = q \frac {\partial E} {\partial l} \vec{l} = \vec{p} \frac {\partial E} {\partial l}
\end{gathered}
\end{equation}
Момент сил относительно центра масс диполя:
\begin{equation}
\begin{gathered}
	\vec{M} = \vec{M_-} + \vec{M_+} = [\vec{r_-} \times \vec{F_-}] + [\vec{r_+} \times \vec{F_+}] = \\
	q ([\vec{r_+} \times \vec{E_+}] - [\vec{r_-} \times \vec{E_-}]) = q [(\vec{r_+} - \vec{r_-}) \times \vec{E}] = \\
	[q \vec{l} \times \vec{E}] = [\vec{p} \times \vec{E}]
\end{gathered}
\end{equation}

Потенциальная энергия диполя: 
\begin{equation}
\begin{gathered}
	W = q \varphi_+ - q \varphi_- = q \Delta \varphi = \\
	q \; \frac{\partial \varphi}{\partial l} l = -p E_l = \\
	-p E \cos {\alpha}
\end{gathered}
\end{equation}


Электронная поляризация возникает в результате смещения электронных облаков относительно центра ядер атомов или ионов под действием электрического поля. Наблюдается во всех без исключения диэлектриках, а в неполярных материалах является единственным видом поляризации.\\
$\vec{p} = q\vec{l}$ - дипольный момент электрического диполя, $a_e = \chi \varepsilon_0$ - электронная поляризуемость, $\varepsilon$ - диэлектрическая проницаемость, $n$ - число частиц в единице объёма\\
\begin{equation}
	P = a_e \cdot E
\end{equation}
\begin{equation}
	\varepsilon = 1 + \frac{n \cdot a_e}{\varepsilon _0}
\end{equation}
\url{http://ftemk.mpei.ac.ru/foetm/files/foetm_book01/foetm_text105.htm}

Ориентационный тип поляризации характерен для полярных диэлектриков. В отсутствие внешнего электрического поля молекулярные диполи ориентированы случайным образом, так что макроскопический электрический момент диэлектрика равен нулю. Если поместить такой диэлектрик во внешнее электрическое поле, то на молекулу-диполь будет действовать момент сил, стремящийся ориентировать ее дипольный момент в направлении напряженности поля. Однако полной ориентации не происходит, поскольку тепловое движение стремится разрушить действие внешнего электрического поля.\\
\begin{equation}
	\vec{P} = n <\vec{p}>
\end{equation}
Поляризованность в этом случае равна $P$, где $<p>$ - среднее значение составляющей дипольного момента молекулы в направлении внешнего поля.\\
\url{http://physicsleti.narod.ru/fiz/html/point_2_2.html}

\subsection{8   Векторы поляризации и электрической индукции и их использование для описания электрического поля при наличии диэлектриков. Примеры вычисления полей}
Вектор поляризации:\\
\begin{equation}
	P_n = \vec{P} \cdot \vec{n}
\end{equation}
$\vec{n}$ - нормаль.\\
Теорема Гаусса для вектора поляризации:
\begin{equation}
	\oint\limits_S {(\vec{P} \cdot d \vec{S})} = -q'
\end{equation}
где $ q' $ - полный связанный заряд системы.
Электрическая индукция:\\
\begin{equation}
	\vec{D} = \varepsilon _0 \vec{E} + \vec{P}
\end{equation}

\subsection{9   Электрическое поле однородно поляризованного шара. Формула Клаузиуса-Моссотти. Спонтанная поляризация диэлектриков.}
Электрическое поле однородно поляризованного шара. \\
Внутри шара: \\
\begin{equation}
        \vec{E} = - \frac{4 \pi}{3} \vec{P}
\end{equation}
За пределами шара:
\begin{equation}
        \vec{E} = - \frac{4 \pi k}{3} \vec{P} \sqrt{1 - 3 \cos^2 {\alpha}}
\end{equation}
\url{http://alexandr4784.narod.ru/sdvepdf3/segl01_16.pdf} \\
Формула Клаузиуса-Мосотти:\\
\begin{equation}
	\frac{\varepsilon - 1}{\varepsilon + 2} = \frac{4 \pi}{3} N \cdot \alpha
\end{equation}
$\varepsilon$ — диэлектрическая проницаемость, $N$ — количество частиц в единице объёма, $\alpha$ — их поляризуемость.\\
Опять же, не нашла ничего толкового по спонтанной поляризации\\
\url{http://ftemk.mpei.ac.ru/ctlw/DocHandler.aspx?p=pubs/etm_full/polarf/02.05.06.htm}

\subsection{10   Электростатическая энергия системы электрических зарядов в вакууме}
Энергия взаимодействия двух зарядов:\\
\begin{equation}
	W = \frac{k \cdot q_1 \cdot q_2}{r}
\end{equation}
Энергия взаимодействия системы зарядов:\\
\begin{equation}
	W = \frac{1}{2} \sum _i q_i \sum _{j \neq i} k \frac{q_j}{r_{ij}} = \frac{1}{2} \sum _{i = 1} ^N q_i \cdot \varphi _i
\end{equation}
\url{http://www.effects.ru/science/278/index.htm}

\subsection{11	Объемная плотность энергии электростатического поля в вакууме}
В общем случае:
\begin{equation}
	\omega = \frac {W}{V} = \frac {\frac{1}{2} \sum _i q_i \sum _{j \neq i} k \frac{q_j}{r_{ij}}} {V}	
\end{equation}
Для плоского конденсатора:
\begin{equation}
	\omega = \frac{\varepsilon \cdot \varepsilon _0 \cdot E^2}{2} = \frac{D^2}{2 \cdot \varepsilon \cdot \varepsilon _0}
\end{equation}
\url{http://physics-lectures.ru/elektrostatika/16-3-obemnaya-plotnost-energii-elektrostaticheskogo-polya/}

\subsection{12   Силы, действующие на диэлектрик в неоднородном электрическом поле}
//Может, это пондеромоторные силы? Он про них точно что-то рассказывал


\section{Электрический ток}
\subsection{1   Основные определения. Закон сохранения электрического заряда. Дифференциальная форма закона Ома для сред с эффективной силой вязкого трения. Объемные токи. Закон Джоуля-Ленца.}
\textbf{Электрический ток} – упорядоченное движение зарядов. Электрический ток может быть обусловлен движением как положительными так и отрицательными зарядами. За положительное направление тока принимают направление движения положительных зарядов.\\
Плотность заряда: $\vec{j} = qn<\vec{v}>$\\
Сила электрического тока, протекающего через сечение проводника: $I = \int \vec{j} \cdot d\vec{S} = \frac{dQ}{dt}$\\
Интегральная форма записи закона сохранения заряда: $\frac{dQ_V}{dt} = - \oint \vec{j} \cdot d\vec{S}$\\
Дифференциальная форма ЗСЗ: $\frac{\delta \rho}{\delta t} = - \nabla \vec{j}$\\
Закон Джоуля-Ленца(из вики): $\omega = \vec{j} \cdot E = \sigma \cdot E^2; dQ = I^2Rdt$, где $\sigma$ - проводимость среды.


\subsection{2	Элементы классической теории проводимости металлов. Дифференциальная и интегральная формы закона Ома для пассивного и активного участков цепи. Удельное сопротивление и его зависимость от температуры. Трудности классической теории проводимости металлов.}
Уравнение движения свободных носителей заряда в проводнике в рамках классической теории проводимости: $\frac{d \vec{v}}{dt} = q \vec{E} + \vec{F} - \eta \vec{v}$\\
Связь коэффициента трения и постоянной времени релаксации: $<v(t)> = v_0 e^{-t/\tau}$\\
Закон Ома в дифференциальной форме: $\vec{j} = \sigma \cot  \vec{E}$\\
$j$ - плотность тока, $\sigma = \frac{1}{\rho}$ - проводимость, $\rho$ - удельное сопротивление\\
Закон Ома в интегральной форме для пассивного участка цепи: $I = \frac{U}{R}$\\
Закон Ома в интегральной форме для активного участка цепи: $I = \frac{-\Delta \varphi \pm \Xi}{R}$, где $\Xi$ - ЭДС\\ 
Правила Кирхгофа:\\
- Сумма токов в любом узле = 0;\\
- Сумма падений напряжений в любом замкнутом контуре = сумме ЭДС.\\
Оценка удельного сопротивления проводника: $\rho = \frac{1}{nq^2} \sqrt{\frac{mkT}{\lambda}}$\\


\subsection{3	Электрический ток в вакууме. Вакуумный диод. (Закон 3/2). Выпрямление электрического тока.}
Оценка величины запирающего потенциала для вакуумного диода: $-e\cdot U_z \approx kT - A$\\
Оценка величины тока насыщения: $I_N = const \cdot e^{-A/kT}$\\
Сила тока, протекающего через вакуумный диод: $I = q\cdot n \cdot v \cdot S$\\
Связь скорости электронов с потенциалом внутри диода: $\frac{mv^2}{2} = q \varphi$\\
Уравнение Пуассона для потенциала в вакуумном диоде: $\frac{d^2\varphi (x)}{dx^2} = - 4\pi \rho = 4 \pi q n (x)$\\
Закон "3/2": $I = const \cdot \varphi ^{3/2}$\\

\subsection{4-6}
Не знаю

\section{Магнитостатика}
\subsection{2-3  Сила Лоренца. Магнитное поле системы движущихся зарядов. Примеры вычисления магнитных полей простейших распределений. Закон Био-Савара-Лапласа. Сила Ампера, момент сил, действующий на магнитный диполь в магнитном поле. Варианты определения вектора B в классических курсах электродинамики}
Магнитное поле, создаваемое в точек R точечным зарядом q, движущимся в точке r с постоянной скоростью v: $\vec{B}(R) = q \frac{[\frac{\vec{v}}{c} , (\vec{R} - \vec{r})]}{|\vec{R} - \vec{r}|^3}$\\
Сила Лоренца: $F_L = Q \left[ \frac{\vec{v}}{c}, \vec{B} \right]$\\
Сила Ампера, действующая на небольшой участок провода: $dF_A  = \frac{l}{c}[d \vec{l}, \vec{B}]$\\
Величина момента сил, действующих на рамку с током $M = \frac{I}{c}SB \sin{\alpha}$\\
Магнитный момент рамки с током: $m = \frac{I}{c}S$\\
Вращающий момент, действующий на магнитный диполь: $\vec{M} = [\vec{m}, \vec{B}]$\\
Магнитное поле, создаваемое системой движущихся зарядов: $\vec{B}(\vec{R}) = \sum _k q_k \left[ \frac{v_k}{c}, \frac{\vec{R} - \vec{r_k}}{|\vec{R} - \vec{r_k}|^3} \right]$\\
Магнитное поле создаваемое объёмным распределением тока в пространстве: $\vec{B}(\vec{R}) = \int dV(\vec{r}) \left[ \frac{\vec{j}(\vec{r})}{c}, \frac{\vec{R} - \vec{r_k}}{|\vec{R} - \vec{r_k}|^3} \right]$\\
Закон Био-Савара-Лапласа: $\vec{B}(\vec{R}) = \frac{I}{c}\int \left[ d\vec{l}, \frac{\vec{R} - \vec{r_k}}{|\vec{R} - \vec{r_k}|^3} \right]$\\

\subsection{Дифференциальная и интегральная формы записи уравнений магнитного поля в вакууме и их использование для решения задач  магнитостатики}
Источники потенциального магнитного поля отсутствуют: $\nabla \cdot \vec{B} = 0$\\
Интегральный аналог того же: $\oint (\vec{B} \cdot d\vec{S}) = 0$\\
Вихревое магнитное поле создаётся электрическими токами: $\nabla \times \vec{B} = \frac{4 \pi \vec{j}}{c}$\\
Циркуляция вектора \textbf{B}: $\int \vec{B} \cdot d\vec{l} = 4 \pi \frac{I}{c}$\\

\subsection{10	Векторы намагниченности и напряженности магнитного поля  и их использование для  записи макроскопических уравнений для магнитостатического поля в веществе. Соленоид с сердечником} 
Векторы поляризации и намагниченности: $\vec{P} = n <\vec{d}>; \vec{M} = n<\vec{m}>$\\




\end{document}
